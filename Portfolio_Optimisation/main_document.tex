\documentclass{article}
\usepackage{lscape}  % To rotate the page
\usepackage{booktabs}  % For better table formatting
\usepackage{geometry}  % To adjust margins if necessary
\usepackage{adjustbox}  % To resize the table if needed
\usepackage{setspace}
\usepackage{longtable}
\usepackage{hyperref}
\usepackage{amsmath}

\title{Portfolio Optimisation}
\author{Lajos Galambos}
\date{October 2024}

\begin{document}
\setstretch{1.5} 
\maketitle

\noindent Check out my \href{https://github.com/galamboslajos/Finance-Working-Files/blob/main/Portfolio_optimization_US/Portfolio_optimization_US.md}{GitHub for Python code}.

\section{Introduction}

The motivation for this project is to effectively manage a \$500,000 investment by identifying optimal, financially sound portfolio strategies. Using Modern Portfolio Theory (MPT), I aim to balance risk and return, constructing portfolios that either maximize returns for a given risk or minimize risk for a target return. This analysis focuses on a selection of U.S. stocks, using 10 years of historical data (2014–2024) to construct three optimized portfolios: the Minimum Variance Portfolio, the Tangency Portfolio (which maximizes the Sharpe ratio), and the Maximum Return Portfolio. Each portfolio is analyzed for its risk-return performance and compared to the S\&P 500 benchmark. While portfolio optimization follows a top-down approach, which can sometimes overlook asset-specific details, it remains a useful tool for managing diversified investments and aligning portfolios with an investor's risk tolerance.



\section{Stocks and Data}

The stocks listed come from a variety of sectors, primarily large-cap companies. There all together 50 US stocks collected in the sample. These companies represent industries such as technology, healthcare, energy, financial, consumer staples, and discretionary. They are generally considered large-cap, industry-leading firms. Figure  \ref{fig:Piechart_Industries} breaks down the distribution of the selected stocks in industries.

In this analysis, I define a time period spanning the last ten years, from January 1, 2014, to January 1, 2024. Using this time frame, I downloaded historical stock data, specifically the adjusted closing prices, for a set of selected U.S. stocks and the S\&P 500 index, which serves as benchmark. The adjusted closing prices are used as they account for corporate actions such as dividends and stock splits, providing a more accurate reflection of the stock's value over time. The data was retrieved using the \texttt{yfinance} library, a Python interface for downloading financial data from \href{https://finance.yahoo.com/}{Yahoo Finance}. This dataset will serve as the basis for portfolio analysis and optimization over the selected period.
 
\begin{figure}[h]
    \centering
    \includegraphics[width=0.8\linewidth]{Piechart_Insudtries.png} 
    \caption{Distribution of the 50 selected US stocks by industry. An extensive list of the 50 companies is provided in the \hyperref[sec:Appendix]{Appendix} section.}
    \label{fig:Piechart_Industries} 
\end{figure}

\section{Methodology: Portfolio Optimization}

This section outlines the methodology used for the portfolio optimization process, implemented in Python. The process includes obtaining historical stock price data, calculating key statistics, optimizing different portfolios, and visualizing the results.

\subsection{Step 1: Data Collection and Return Computation}
I retrieved daily adjusted closing prices for the selected stocks and the benchmark index (S\&P 500) from Yahoo Finance using Python. The daily returns were calculated as:

\begin{equation}
r_{t,i} = \frac{P_{t,i} - P_{t-1,i}}{P_{t-1,i}}
\end{equation}

where $r_{t,i}$ is the return for stock $i$ at time $t$ and $P_{t,i}$ is the adjusted closing price at time $t$. Using these daily returns, I computed the annualized returns and volatilities for each stock:

\begin{equation}
\mu_i = \bar{r_i} \times 252, \quad \sigma_i = \sigma_{\text{daily},i} \times \sqrt{252}
\end{equation}

where $\bar{r_i}$ is the average daily return, $\sigma_{\text{daily},i}$ is the standard deviation of daily returns, and 252 represents the number of trading days in a year.

\subsection{Step 2: Covariance Matrix and Portfolio Performance}
The covariance matrix of the stock returns was computed to account for the relationships between stocks:

\begin{equation}
\Sigma_{ij} = \text{Cov}(r_i, r_j)
\end{equation}

Given the returns, volatilities, and covariance matrix, the portfolio's expected return and volatility were calculated as:

\begin{equation}
\mu_p = \sum_{i=1}^{n} w_i \mu_i, \quad \sigma_p = \sqrt{\mathbf{w}^T \Sigma \mathbf{w}}
\end{equation}

where $\mathbf{w}$ is the vector of portfolio weights, $\mu_i$ is the expected return of stock $i$, and $\Sigma$ is the covariance matrix of returns.

\subsection{Step 3: Portfolio Optimization}
I performed optimization for three types of portfolios:

\paragraph{a) Minimum Variance Portfolio:} This portfolio minimizes the overall portfolio risk (volatility) subject to the constraint that the sum of the portfolio weights equals 1. The objective is to minimize:

\begin{equation}
\text{Minimize: } \sigma_p = \sqrt{\mathbf{w}^T \Sigma \mathbf{w}}
\end{equation}

\paragraph{b) Tangency Portfolio (Maximum Sharpe Ratio Portfolio):} The tangency portfolio maximizes the Sharpe ratio, defined as:

\begin{equation}
\text{Sharpe Ratio} = \frac{\mu_p - r_f}{\sigma_p}
\end{equation}

where $r_f$ is the risk-free rate. This portfolio represents the optimal mix of risky assets.

\paragraph{c) Maximum Return Portfolio:} The portfolio that assigns all weights to the stock with the highest expected return.

\subsection{Step 4: Efficient Frontier and Capital Market Line (CML)}
I generated a large number of random portfolios to visualize the efficient frontier, which represents the set of portfolios offering the maximum return for a given level of risk. The portfolios are scattered along this curve, and key portfolios such as the minimum variance and tangency portfolios are highlighted.

The Capital Market Line (CML) is the straight line that originates from the risk-free rate and is tangent to the efficient frontier at the tangency portfolio. It is calculated as:

\begin{equation}
E(R_p) = r_f + \frac{E(R_m) - r_f}{\sigma_m} \times \sigma_p
\end{equation}

where $E(R_p)$ is the expected return of the portfolio, $r_f$ is the risk-free rate, $E(R_m)$ is the expected return of the tangency portfolio, and $\sigma_p$ is the portfolio's risk.

\section{Results}

\subsection{Evaluation of the Efficient Frontier, Optimal Portfolios, and Capital Market Line}

Figure \ref{fig:Efficiency_Frontier} illustrates the efficient frontier along with three key portfolios: the Minimum Variance Portfolio, Tangency Portfolio, and Maximum Return Portfolio. Additionally, the Capital Market Line (CML) is plotted to demonstrate the optimal risk-return combinations, factoring in the risk-free rate. 

\begin{figure}[h]
    \centering
    \includegraphics[width=1\linewidth]{Efficiency_Frontier.png}  % Specify the image file and width
    \caption{Efficiency Frontier}
    \label{fig:Efficiency_Frontier} 
\end{figure}

\begin{itemize}
    \item \textbf{Efficient Frontier:} The scatter plot of random portfolios forms the efficient frontier, representing portfolios that offer the highest possible return for a given level of risk. As risk (annualized volatility) increases, the potential return also increases. The efficient frontier is plotted using 100,000 randomly generated portfolios, ensuring a wide coverage of the risk-return space. The concentration of points near the lower left suggests that many portfolios exist with relatively low risk and return, while only a few portfolios offer extremely high returns with high risk.
    
    \item \textbf{Minimum Variance Portfolio (Red Star):} This portfolio has the lowest volatility among all possible portfolios. It lies at the extreme left of the efficient frontier, indicating that it has the smallest annualized risk. However, this also corresponds to a relatively low annualized return.
    
    \item \textbf{Tangency Portfolio (Blue Star):} The tangency portfolio is where the Capital Market Line (CML) touches the efficient frontier. This portfolio maximizes the Sharpe ratio, balancing risk and return most efficiently. As seen in the graph, the tangency portfolio provides a higher return than the minimum variance portfolio for a higher risk.
    
    \item \textbf{Maximum Return Portfolio (Green Star):} The maximum return portfolio assigns all the weight to the stock with the highest expected return, leading to the highest possible return at the expense of extremely high volatility. This portfolio lies at the far right of the frontier and below the CML, demonstrating the trade-off between return and risk.
    
    \item \textbf{Benchmark (S\&P 500, Black Cross):} The benchmark, represented by the S\&P 500 index, lies relatively close to the efficient frontier but does not necessarily outperform optimized portfolios in terms of risk-adjusted returns. The benchmark has a slightly higher risk compared to the minimum variance portfolio and around the same returns.
    
    \item \textbf{Capital Market Line (CML, Orange Dashed Line):} The CML is the straight line that starts from the risk-free rate (assumed 3\%) and is tangent to the efficient frontier at the tangency portfolio. It represents portfolios that combine the risk-free asset with risky assets, offering better risk-adjusted returns than any portfolio lying below the CML. As expected, the slope of the CML reflects the Sharpe ratio of the tangency portfolio, which is the maximum attainable.
    
    \item \textbf{Sharpe Ratio Heatmap:} The color map represents the Sharpe ratios of the portfolios, with warmer colors indicating higher Sharpe ratios. The tangency portfolio, located where the CML touches the efficient frontier, maximizes the Sharpe ratio, which explains why it is located in an area with the highest concentration of warm colors (yellow-green).
\end{itemize}

In conclusion, the graph illustrates the trade-off between risk and return, with the CML serving as a guide for the optimal portfolio selection in the presence of a risk-free asset. Portfolios that lie on the CML dominate those on the efficient frontier in terms of risk-adjusted returns.

\subsection{Evaluation of Portfolio Weights (Table 1)}

Table 1 summarizes the portfolio weights for three distinct portfolios—Minimum Variance, Tangency, and Maximum Return. Below is the evaluation of each portfolio:

\begin{table}[h]
    \centering
    \begin{adjustbox}{width=0.8\linewidth,center}  % Scale table to improve readability and fit within the page width
      \begin{tabular}{llll}
\toprule
Asset & Min Variance Weights & Tangency Weights & Max Return Weights \\
\midrule
AAPL & 0.000000 & 0.264700 & 0.000000 \\
MSFT & 0.035500 & 0.000000 & 0.000000 \\
V & 0.022700 & 0.000000 & 0.000000 \\
WMT & 0.217900 & 0.000000 & 0.000000 \\
PG & 0.000000 & 0.009200 & 0.000000 \\
XOM & 0.220900 & 0.000000 & 0.000000 \\
BAC & 0.000000 & 0.426800 & 0.000000 \\
KO & 0.013100 & 0.091200 & 0.000000 \\
PFE & 0.093400 & 0.000000 & 0.000000 \\
CSCO & 0.153600 & 0.000000 & 0.000000 \\
NFLX & 0.000000 & 0.178200 & 0.000000 \\
NKE & 0.185400 & 0.029800 & 0.000000 \\
C & 0.057600 & 0.000000 & 0.000000 \\
MDT & 0.000000 & 0.000000 & 1.000000 \\
\midrule
Return & 0.1460 & 0.6102 & 0.6641 \\
Volatility & 0.1569 & 0.2396 & 0.2655 \\
Sharpe Ratio & 0.9305 & 2.5470 & 2.5009 \\
\bottomrule
\end{tabular}
    \end{adjustbox}
    \caption{Portfolio Weights and Statistics for Min Variance, Tangency, and Max Return Portfolios}
\end{table}


\begin{itemize}
\item\textbf{Minimum Variance Portfolio:} 
     The objective of this portfolio is to minimize risk by reducing overall portfolio volatility. The key allocations are focused on lower-risk, stable stocks, with significant weights in companies like \textbf{Walmart (21.79\%)}, \textbf{Exxon Mobil (22.09\%)}, and \textbf{Cisco Systems (15.36\%)}. These allocations reflect a focus on sectors like consumer staples and energy, which are typically less volatile. The portfolio achieves a return of \textbf{14.60\%} with a volatility of \textbf{15.69\%}, resulting in a Sharpe ratio of \textbf{0.93}. This balanced risk-return trade-off makes the portfolio ideal for investors seeking lower volatility while still achieving reasonable returns.
\end{itemize}


\begin{itemize}
    \item\textbf{Tangency Portfolio:}  This portfolio aims to maximize the risk-adjusted return by achieving the highest possible Sharpe ratio. The largest allocations are made to high-performing stocks such as \textbf{Apple (26.47\%)} and \textbf{Bank of America (42.68\%)}, highlighting a focus on both technology and financial sectors. The portfolio produces an impressive return of \textbf{61.02\%} with a volatility of \textbf{23.96\%}, yielding an outstanding Sharpe ratio of \textbf{2.55}. This shows that the portfolio delivers substantial returns for each unit of risk taken. It's the best choice for investors prioritizing risk-adjusted returns over absolute returns.
\end{itemize}

\begin{itemize}
    \item \textbf{Maximum Return Portfolio:} This portfolio's sole objective is to maximize returns, regardless of the associated risk. It is fully concentrated in \textbf{Medtronic (100\%)}, a leading healthcare company. By investing entirely in this single stock, the portfolio achieves the highest return of \textbf{66.41\%}, but it also comes with the highest volatility at \textbf{26.55\%}. The Sharpe ratio is \textbf{2.50}, which, while impressive, reflects a more concentrated risk compared to the tangency portfolio. This portfolio is suitable for aggressive investors willing to take on concentrated risk in pursuit of the highest possible returns.
\end{itemize}


Each portfolio offers unique benefits: the Minimum Variance Portfolio provides risk minimization, the Tangency Portfolio offers the best risk-adjusted return, and the Maximum Return Portfolio is focused on maximizing returns with higher concentration of risk.

\subsection{Performance of Tangency Weight in the Past}

\textbf{Figure 3} (Portfolio Performance vs S\&P 500 for the Last Year) and \textbf{Figure 4} (Portfolio Performance vs S\&P 500 for the Last 5 Years) illustrate the cumulative performance of the Min Variance, Tangency, and Max Return portfolios compared to the S\&P 500 benchmark. In Figure 3, the Max Return Portfolio exhibits the highest returns but with notable volatility, reaching approximately 2.4 times the initial investment. The Tangency Portfolio, designed for optimal risk-adjusted returns, tracks the S\&P 500 closely but with less volatility, delivering a higher return than the benchmark. The Min Variance Portfolio, focused on minimizing volatility, maintains stability but underperforms compared to the other portfolios and the benchmark. Figure 4, covering a 5-year period, shows the Max Return Portfolio delivering the highest returns over time with greater volatility, while the Tangency Portfolio consistently outperforms the S\&P 500 with a more balanced risk-return profile. The Min Variance Portfolio remains the most stable but lags behind in returns, highlighting its lower-risk strategy over the long term.

\begin{figure}[!h]  % Force LaTeX to place the figures together on the same page
    \centering

    \begin{minipage}{\textwidth}
        \centering
        \includegraphics[width=0.9\textwidth]{Portfolio_Performance.png}
        \caption{Performance of Portfolios and the Benchmark (1 Year)}
        \label{fig:Portfolio_Performance}
    \end{minipage}
    
    \vspace{1cm}  % Vertical space between the two figures

    \begin{minipage}{\textwidth}
        \centering
        \includegraphics[width=0.9\textwidth]{Portfolio_Performance_Long.png}
        \caption{Performance of Portfolios and the Benchmark (5 Years)}
        \label{fig:Portfolio_Performance_Long}
    \end{minipage}

\end{figure}

\clearpage  % Ensure that the Conclusion section starts on a new page

\section{Conclusion}

In this analysis, I presented an attempt to efficiently invest \$500.000 based on the top-down method of portfolio analysis. I evaluated the performance of three distinct portfolios: the Minimum Variance Portfolio, the Tangency Portfolio, and the Maximum Return Portfolio, alongside the S\&P 500 benchmark. By utilizing a historical dataset of 50 large-cap US stocks over the last five years, I optimized each portfolio to meet specific investment objectives, such as minimizing volatility or maximizing risk-adjusted returns. The results show that the Tangency Portfolio consistently outperforms the benchmark in terms of the Sharpe ratio, reflecting its superior risk-adjusted performance. The Maximum Return Portfolio achieves the highest returns but with significantly higher volatility, while the Minimum Variance Portfolio offers a stable, lower-risk option. These findings highlight the trade-offs investors must consider between risk and return when constructing optimal portfolios.

\newpage  % Ensure that the Appendix starts on a new page

\section{Appendix}



In the Appendix, we provide two important tables. The first table shows the portfolio weights for the Min Variance, Tangency, and Max Return portfolios across the 50 selected US stocks. This table also includes key performance metrics for each portfolio, such as return, volatility, and Sharpe ratio, which help compare the risk-return profiles. The second table  categorizes the same 50 stocks by sector, providing insights into the industry distribution across sectors like Technology, Consumer Discretionary, Financial, and Healthcare, illustrating the diverse composition of the selected companies.

 
\begin{landscape}

\begin{table}
    \centering
    \begin{adjustbox}{width=\linewidth,center}  % Scale table to fit the entire page width and center horizontally
      \begin{tabular}{llll}
\toprule
Asset & Min Variance Weights & Tangency Weights & Max Return Weights \\
\midrule
AAPL & 0.000000 & 0.264700 & 0.000000 \\
MSFT & 0.035500 & 0.000000 & 0.000000 \\
GOOGL & 0.000000 & 0.000000 & 0.000000 \\
AMZN & 0.000000 & 0.000000 & 0.000000 \\
TSLA & 0.000000 & 0.000000 & 0.000000 \\
JPM & 0.000000 & 0.000000 & 0.000000 \\
V & 0.022700 & 0.000000 & 0.000000 \\
JNJ & 0.000000 & 0.000000 & 0.000000 \\
WMT & 0.217900 & 0.000000 & 0.000000 \\
PG & 0.000000 & 0.009200 & 0.000000 \\
XOM & 0.220900 & 0.000000 & 0.000000 \\
BAC & 0.000000 & 0.426800 & 0.000000 \\
KO & 0.013100 & 0.091200 & 0.000000 \\
DIS & 0.000000 & 0.000000 & 0.000000 \\
PFE & 0.093400 & 0.000000 & 0.000000 \\
CSCO & 0.153600 & 0.000000 & 0.000000 \\
NFLX & 0.000000 & 0.178200 & 0.000000 \\
BA & 0.000000 & 0.000000 & 0.000000 \\
NKE & 0.185400 & 0.029800 & 0.000000 \\
C & 0.057600 & 0.000000 & 0.000000 \\
INTC & 0.000000 & 0.000000 & 0.000000 \\
IBM & 0.000000 & 0.000000 & 0.000000 \\
NVDA & 0.000000 & 0.000000 & 0.000000 \\
ADBE & 0.000000 & 0.000000 & 0.000000 \\
PYPL & 0.000000 & 0.000000 & 0.000000 \\
MA & 0.000000 & 0.000000 & 0.000000 \\
MRK & 0.000000 & 0.000000 & 0.000000 \\
PEP & 0.000000 & 0.000000 & 0.000000 \\
MCD & 0.000000 & 0.000000 & 0.000000 \\
ABT & 0.000000 & 0.000000 & 0.000000 \\
CRM & 0.000000 & 0.000000 & 0.000000 \\
ORCL & 0.000000 & 0.000000 & 0.000000 \\
CVX & 0.000000 & 0.000000 & 0.000000 \\
T & 0.000000 & 0.000000 & 0.000000 \\
MDT & 0.000000 & 0.000000 & 1.000000 \\
HON & 0.000000 & 0.000000 & 0.000000 \\
GE & 0.000000 & 0.000000 & 0.000000 \\
MMM & 0.000000 & 0.000000 & 0.000000 \\
CAT & 0.000000 & 0.000000 & 0.000000 \\
LOW & 0.000000 & 0.000000 & 0.000000 \\
LMT & 0.000000 & 0.000000 & 0.000000 \\
UNH & 0.000000 & 0.000000 & 0.000000 \\
HD & 0.000000 & 0.000000 & 0.000000 \\
TXN & 0.000000 & 0.000000 & 0.000000 \\
GS & 0.000000 & 0.000000 & 0.000000 \\
AXP & 0.000000 & 0.000000 & 0.000000 \\
SBUX & 0.000000 & 0.000000 & 0.000000 \\
UPS & 0.000000 & 0.000000 & 0.000000 \\
FDX & 0.000000 & 0.000000 & 0.000000 \\
TGT & 0.000000 & 0.000000 & 0.000000 \\
Portfolio Stats & Return: 0.1460, Volatility: 0.1569, Sharpe: 0.9305 & Return: 0.6102, Volatility: 0.2396, Sharpe: 2.5470 & Return: 0.6641, Volatility: 0.2655, Sharpe: 2.5009 \\
\bottomrule
\end{tabular}
    \end{adjustbox}
    \caption{Portfolio Weights and Statistics for Min Variance, Tangency, and Max Return Portfolios}
\end{table}

\end{landscape}

\begin{landscape}

\begin{center}
\begin{adjustbox}{width=\linewidth}
\begin{tabular}{p{3.5cm} p{3.5cm} p{3.5cm} p{3.5cm} p{3.5cm} p{3.5cm}}  % Adjust column widths
\toprule
\textbf{Technology} & \textbf{Consumer Discretionary} & \textbf{Financials} & \textbf{Healthcare} & \textbf{Industrials} & \textbf{Energy / Consumer Staples / Communication Services} \\
\midrule
\textbf{AAPL} - Apple Inc. & \textbf{AMZN} - Amazon.com, Inc. & \textbf{JPM} - JPMorgan Chase & \textbf{JNJ} - Johnson \& Johnson & \textbf{BA} - Boeing Co. & \textbf{XOM} - Exxon Mobil \\
\textbf{MSFT} - Microsoft Corp. & \textbf{TSLA} - Tesla, Inc. & \textbf{V} - Visa Inc. & \textbf{PFE} - Pfizer Inc. & \textbf{HON} - Honeywell Int'l & \textbf{CVX} - Chevron Corp. \\
\textbf{GOOGL} - Alphabet Inc. & \textbf{NKE} - Nike, Inc. & \textbf{BAC} - Bank of America & \textbf{MRK} - Merck \& Co. & \textbf{GE} - General Electric & \textbf{PG} - Procter \& Gamble \\
\textbf{CSCO} - Cisco Systems & \textbf{WMT} - Walmart Inc. & \textbf{C} - Citigroup Inc. & \textbf{ABT} - Abbott Labs & \textbf{MMM} - 3M Company & \textbf{KO} - Coca-Cola Co. \\
\textbf{NFLX} - Netflix, Inc. & \textbf{MCD} - McDonald's Corp. & \textbf{PYPL} - PayPal Holdings & \textbf{MDT} - Medtronic plc & \textbf{CAT} - Caterpillar Inc. & \textbf{DIS} - Walt Disney \\
\textbf{INTC} - Intel Corp. & \textbf{LOW} - Lowe's Companies & \textbf{MA} - Mastercard Inc. & \textbf{UNH} - UnitedHealth Group & \textbf{LMT} - Lockheed Martin & \textbf{T} - AT\&T Inc. \\
\textbf{IBM} - IBM Corp. & \textbf{HD} - The Home Depot & \textbf{GS} - Goldman Sachs &  & \textbf{UPS} - United Parcel Service & \textbf{PEP} - PepsiCo Inc. \\
\textbf{NVDA} - NVIDIA Corp. & \textbf{SBUX} - Starbucks Corp. & \textbf{AXP} - American Express &  & \textbf{FDX} - FedEx Corporation &  \\
\textbf{ADBE} - Adobe Inc. & \textbf{TGT} - Target Corp. &  &  &  &  \\
\textbf{CRM} - Salesforce, Inc. &  &  &  &  &  \\
\textbf{ORCL} - Oracle Corp. &  &  &  &  &  \\
\textbf{TXN} - Texas Instruments &  &  &  &  &  \\
\bottomrule
\end{tabular}
\end{adjustbox}
\caption{List of US publicly listed companies in the sample of 50 by industry}
\end{center}

\end{landscape}

\end{document}





