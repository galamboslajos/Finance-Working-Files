\documentclass{beamer}
\usepackage{lscape}
\usepackage{booktabs}
\usepackage{geometry}
\usepackage{adjustbox}
\usepackage{setspace}
\usepackage{longtable}
\usepackage{hyperref}
\usepackage{amsmath}

\title{Portfolio Optimisation}
\author{Lajos Galambos}
\date{October 2024}

\begin{document}

\frame{\titlepage}

\begin{frame}
\frametitle{Introduction}
The motivation for this project is to effectively manage a \$500,000 investment by identifying optimal, financially sound portfolio strategies (full project:\href{https://github.com/galamboslajos/Finance-Working-Files/blob/main/Portfolio_optimization_US}{GitHub}).
\begin{itemize}
    \item Use of Modern Portfolio Theory (MPT) to balance risk and return.
    \item Focus on three portfolios: Min Variance, Tangency, and Max Return.
    \item Historical data (2014–2024) on 50 US large-cap stocks.
    \item Benchmark: S\&P 500 index.
\end{itemize}
\end{frame}

\begin{frame}
\frametitle{Stocks and Data}
\begin{itemize}
    \item 50 US large-cap stocks from various sectors (technology, healthcare, energy, etc.).
    \item Data: Adjusted closing prices (Jan 2014–Jan 2024).
    \item Data source: Yahoo Finance via \texttt{yfinance} Python library.
\end{itemize}
\begin{figure}
    \centering
    \includegraphics[width=0.6\linewidth]{Piechart_Insudtries.png}  % Resized to 60% width
    \caption{Distribution of the 50 selected US stocks by industry.}
\end{figure}
\end{frame}

\begin{frame}
\frametitle{Methodology: Portfolio Optimization}
\begin{enumerate}
    \item Data Collection and Return Computation.
    \item Covariance Matrix and Portfolio Performance.
    \item Portfolio Optimization for Min Variance, Tangency, and Max Return portfolios.
\end{enumerate}
\end{frame}

\begin{frame}
\frametitle{Sharpe Ratio}
The Sharpe ratio is used to measure the risk-adjusted return of a portfolio and is defined as:
\begin{equation}
\text{Sharpe Ratio} = \frac{\mu_p - r_f}{\sigma_p}
\end{equation}
where:
\begin{itemize}
    \item $\mu_p$ = Expected portfolio return,
    \item $r_f$ = Risk-free rate,
    \item $\sigma_p$ = Portfolio volatility.
\end{itemize}
\end{frame}

\begin{frame}
\frametitle{Portfolio Optimization Types}
\begin{itemize}
    \item Minimum Variance Portfolio: Minimizes risk.
    \item Tangency Portfolio: Maximizes Sharpe ratio.
    \item Maximum Return Portfolio: Assigns all weight to the highest-returning stock.
\end{itemize}
\end{frame}

\begin{frame}
\frametitle{Efficient Frontier and CML}
\begin{itemize}
    \item Efficient frontier: Shows optimal portfolios at different risk levels.
    \item Capital Market Line (CML): Reflects risk-free rate + optimal portfolio mix.
\end{itemize}
\begin{figure}
    \centering
    \includegraphics[width=0.9\linewidth]{Efficiency_Frontier.png}
    \caption{Efficient Frontier and Capital Market Line (CML).}
\end{figure}
\end{frame}

\begin{frame}
\frametitle{Portfolio Weights}
\begin{table}[h]
    \centering
    \begin{adjustbox}{width=0.8\linewidth,center}  % Scale table to improve readability and fit within the page width
      \begin{tabular}{llll}
\toprule
Asset & Min Variance Weights & Tangency Weights & Max Return Weights \\
\midrule
AAPL & 0.000000 & 0.264700 & 0.000000 \\
MSFT & 0.035500 & 0.000000 & 0.000000 \\
V & 0.022700 & 0.000000 & 0.000000 \\
WMT & 0.217900 & 0.000000 & 0.000000 \\
PG & 0.000000 & 0.009200 & 0.000000 \\
XOM & 0.220900 & 0.000000 & 0.000000 \\
BAC & 0.000000 & 0.426800 & 0.000000 \\
KO & 0.013100 & 0.091200 & 0.000000 \\
PFE & 0.093400 & 0.000000 & 0.000000 \\
CSCO & 0.153600 & 0.000000 & 0.000000 \\
NFLX & 0.000000 & 0.178200 & 0.000000 \\
NKE & 0.185400 & 0.029800 & 0.000000 \\
C & 0.057600 & 0.000000 & 0.000000 \\
MDT & 0.000000 & 0.000000 & 1.000000 \\
\midrule
Return & 0.1460 & 0.6102 & 0.6641 \\
Volatility & 0.1569 & 0.2396 & 0.2655 \\
Sharpe Ratio & 0.9305 & 2.5470 & 2.5009 \\
\bottomrule
\end{tabular}
    \end{adjustbox}
    \caption{Portfolio Weights and Statistics for Min Variance, Tangency, and Max Return Portfolios}
\end{table}
\end{frame}

\begin{frame}
\frametitle{Results}
\begin{itemize}
    \item Min Variance Portfolio: Lower risk, moderate return.
    \item Tangency Portfolio: Best risk-adjusted return.
    \item Max Return Portfolio: Highest return, highest risk.
\end{itemize}
\end{frame}

\begin{frame}
\frametitle{Had we invested \$500,000 five years ago...}
\begin{figure}
    \centering
    \includegraphics[width=0.8\linewidth]{Portfolio_Performance_Long.png}  % Adjust the width as necessary
    \caption{Performance of the \$500,000 investment over the last five years based on different portfolio strategies.}
\end{figure}
\end{frame}

\begin{frame}
\frametitle{Conclusion}
\begin{itemize}
    \item Tangency Portfolio offers the best balance of risk and return.
    \item Min Variance Portfolio is ideal for risk-averse investors.
    \item Max Return Portfolio is for aggressive investors seeking high returns.
\end{itemize}
\end{frame}

\end{document}
