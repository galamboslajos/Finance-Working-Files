\documentclass[serif, aspectratio=169]{beamer}
%\documentclass[serif]{beamer}  % for 4:3 ratio
\usepackage[T1]{fontenc} 
\usepackage{fourier} % see "http://faq.ktug.org/wiki/uploads/MathFonts.pdf" for other options
\usepackage{hyperref}
\usepackage{latexsym,amsmath,xcolor,multicol,booktabs,calligra}
\usepackage{graphicx,pstricks,listings,stackengine}
\usepackage{lipsum}

\author{Lajos Galambos}
\title{Discounted Cashflow Model (DCF)}
\subtitle{A fundamental valuation method}

\date{\small \today}
\usepackage{HKUSTstyle}

% defs
\def\cmd#1{\texttt{\color{red}\footnotesize $\backslash$#1}}
\def\env#1{\texttt{\color{blue}\footnotesize #1}}
% set colors
\definecolor{hkustyellow}{RGB}{167, 131, 55}
\definecolor{hkustblue}{RGB}{0, 56, 116}
\definecolor{hkustred}{RGB}{209, 51, 59}


\lstset{
    basicstyle=\ttfamily\small,
    keywordstyle=\bfseries\color{deepblue},
    emphstyle=\ttfamily\color{deepred},    % Custom highlighting style
    stringstyle=\color{deepgreen},
    numbers=left,
    numberstyle=\small\color{halfgray},
    rulesepcolor=\color{red!20!green!20!blue!20},
    frame=shadowbox,
}

%- --- --- --- --- --- --- --- --- --- --- --- --- --- --- --- 
\begin{document}

\begin{frame}
    \titlepage
    \vspace*{-0.6cm}
\end{frame}

\begin{frame}    
\tableofcontents[sectionstyle=show,
subsectionstyle=show/shaded/hide,
subsubsectionstyle=show/shaded/hide]
\end{frame}

% Introduction --- --- --- --- --- --- --- --- --- --- --- --- 

\section{Introduction}
\begin{frame}{Background}
	\frametitle<presentation>{Background}
	\begin{block}{What is a DCF}
		\begin{itemize}
			\item It is a valuation method.
			\item Broadly, it is a method that is used to estimate the value of an asset based on its \textbf{future cash flows}.
			\item It can be applied for companies, projects, anything that has cash flows and needs a fundamental value.
   \item It is an \textbf{intrinsic valuation} method, which means that the valuation framework is independent from external factors, relies solely on the firm's, project's ability to create cash flows. 
   \item Alternative valuation technique: \textit{Relative Valuation} (to competitors).
		\end{itemize}
	\end{block}
	
\end{frame}

% Discounted Cash flow (steps) --- --- --- --- --- --- --- --- --- --- 
\section{Discounted Cash flow (steps)}
\begin{frame}{Steps needed for a DCF model}
	\begin{enumerate}
			\item Forecast and calculate \textbf{Free Cash Flow}.
			\item Calculate the \textbf{Weighted Average Cost of Capital (WACC)}.
			\item Calculate the \textbf{Terminal Value}.
            \item \textbf{Discount} the Free Cash Flow and the Terminal Value. 
            \item Calculate the \textit{implied} share price.
		\end{enumerate}
	
\end{frame}

\begin{frame}
	\frametitle<presentation>{1. Forecast and calculate Free Cash Flow}
\begin{block}{Free Cash flow (FCF)}
		\begin{itemize}
			\item It is the cash flow available for both debt and equity holders after the business pays for everything it needs to continue its operation.
			\item Payments in that sense mean paying for operating expenses, capital expenditure, investments.
			\item The more Free Cash flow a company has, the more valuable it is, and more attractive it is for investors, because it is able to pay down its debt or/and invest in new business opportunities.
    \end{itemize}
	\end{block}
\end{frame}

\begin{frame}
	\frametitle<presentation>{1. Forecast and calculate Free Cash Flow}
\begin{block}{Free Cash flow formula}
\begin{itemize}
    \item Departing from \textbf{EBIT }(Earnings Before Interest and Taxes\textbf{)}.
    \item After deducting taxes,
    \item and adding back appreciation/depreciation \textit{(because those are not cash flow related items)},
    \item subtracting Capital Expenditures, 
    \item and adding the net change in the Working Capital (\textit{increase is negative}) yields the \textbf{FCF}. 
\end{itemize}
\begin{align*}
\text{\textbf{Free Cash flow}} = & \text{EBIT} \times (1-\text{tax rate}) - \text{Capital Expenditure} \\
                        & + \text{Depreciation/Appreciation} \\
                        & + \text{changes in working capital}
\end{align*}
\end{block}
\end{frame}

\begin{frame}
	\frametitle<presentation>{2. Calculating the Weighted Average Cost of Capital (WACC)}
\begin{block}{WACC}
		\begin{itemize}
			\item The WACC measures the cost of financing for a company. 
			\item Financing can come in the form of \textbf{Debt} and \textbf{Equity}. 
			\item The cost of debt financing is the interest payment, the cost of equity financing comes from the expected return on the stock (Capital Asset Pricing Model).
            \item The expected return on equity must reflect the level of risk that the individual company embodies.
    \end{itemize}
    \begin{equation*}
   E(R_i) = R_f + \beta_i (E(R_m) - R_f)     
    \end{equation*}
	\end{block}
\end{frame}

\begin{frame}
	\frametitle<presentation>{2. Calculating the Weighted Average Cost of Capital (WACC)}
\begin{block}{WACC}
		
    \begin{equation*}
WACC = \frac{E}{V} \times R_e + \frac{D}{V} \times R_d \times (1 - Tax)
    \end{equation*}
    \textit{(E)} is the market value of the firm's equity,
\textit{(V)} is the total market value of both the firm's equity and debt,
\textit{(Re)} is the cost of equity,
\textit{(D)} is the market value of the firm's debt,
\textit{(Rd)} is the cost of debt,
\textit{(Tax)} is the corporate tax rate.
	\end{block}
\end{frame}

\begin{frame}
	\frametitle<presentation>{3. Calculating the Terminal Vaue}
\begin{block}{Terminal Value}
		\begin{itemize}
			\item The Terminal Value is the is the value of the business after the forecasted period. 
			\item Terminal Value uses growth assumptions that goes to infinity \textit{(perpetuity)}.
    \end{itemize}
    \begin{equation*}
TV = \frac{{FCF_n \times (1+g)}}{{WACC - g}}    
    \end{equation*}
    TV is the terminal value
\textit{(FCFn)} is the free cash flow at period \textit{n} (the final forecasted period's FCF),
\textit{g} is the growth rate (often the GDP growth rate is implied),
\textit{WACC} is the weighted average cost of capital.
	\end{block}
\end{frame}

\begin{frame}
	\frametitle<presentation>{4. Discounting of the FCF and Terminal Value}
\begin{block}{Discounting}
		\begin{itemize}
			\item Discounting is needed in order to bring cash flows from different periods to the present and therefore making them additive. 
    \end{itemize}
    \begin{equation*}
PV = \frac{{FCF_{t1}}}{{(1 + WACC)^1}} + \frac{{FCF_{t2}}}{{(1 + WACC)^2}} + \ldots + \frac{{FCF_{tn}}}{{(1 + WACC)^n}} + \frac{{TV}}{{(1 + WACC)^n}}
    \end{equation*}
\textit{(PV)} is the present value
\textit{(FCF{ti})} is the free cash flow at time \textit{(i)}
\textit{(WACC)} is the weighted average cost of capital
\textit{(TV)} is the terminal value
\textit{(n) }is the final period in the forecast.
	\end{block}
\end{frame}

% Equity Value --- --- --- --- --- --- --- --- --- -- --- 
\section{Equity Value}
\begin{frame}{Enterprise Value, Equity Value}
    \begin{itemize}
        \item The Enterprise Value consists of the market value of the Debt and Equity.
        \item Specifically: it is the Net Debt \textit{(Debt-Cash)} and the value of the Equity. 
    \end{itemize}
  \begin{equation*}
    EV = Net\ Debt + Equity = (Debt - Cash) + Equity
    \end{equation*}

      \begin{equation*}
  Implied\ Share\ Price = \frac{{Equity}}{{Number\ of\ Shares}} = \frac{{EV - (Debt - Cash)}}{{Number\ of\ Shares}}
    \end{equation*}
\end{frame}

\begin{frame}
	\frametitle<presentation>{Alternative Valuation Methods}

		\begin{itemize}
			\item \textbf{Dividend Discounting Model} (Gordon Model) values a company based on only the present value of its expected future dividends. 
            \item \textbf{Economic Value Added} (Residual Income Model)  method focuses on the company’s ability to generate returns above its cost of capital. EVA is calculated as the net operating profit after taxes (NOPAT) minus a charge for the capital employed (cost of capital times the capital).
            \item \textbf{Real Option Valuation} considers the value of potential future opportunities or strategic options, such as delaying, expanding, or abandoning projects.
    \end{itemize}

\end{frame}

% --- Thank you slide ---
\begin{frame}
\begin{center}

\vspace{1cm}

Lajos Galambos \\[1em]
galambos.lajos2000@gmail.com
\end{center}
\end{frame}

\end{document}